\section{Introduction}
The reasoning behind this idea is that global phenology, at a very basic (and
possibly incorrect level), operates in simple harmonic motion. The underlying
signal can be simply described as:

\begin{equation}
    \Theta(t) = \Theta_0 \sin\left(\frac{t\pi}{T}\right)
\end{equation}

where $t$ is some arbitrary point in time, $\Theta_0$ is the initial amplitude
or angle from the perpendicular and $T$ is the period (or frequency).

\begin{center}
\begin{tikzpicture}[scale=1.25]
    \draw[red,thick,<->] (-1,1) parabola bend (0,0) (2.1,4.41)
        node[below right] {$y=x^2$};
    \draw[loosely dotted] (-1,0) grid (4,4);
\end{tikzpicture}
\end{center}

Tree and grass phenology can be represented by two pendulums, where both are
coupled to one another via soil water status of the soil layers. 

The amplitude of the pendulum changes at an annual time-step and is linked to
the soil water content of the root zone and volume of incoming rainfall. For
trees, that access deeper soil water stores, the amplitude of the signal is
small, but for grasses, whose rooting zone is limited to the upper 0.5 m, their
amplitude is much higher, such that $\Theta_g > \Theta_t$.

The period $T$ of the signal is related to the growing season and has a
relation to the length of the pendulum by $T=2\pi\sqrt(L/g)$. At this point we
assume mass plays no role in the motion of the pendulum (untrue as we could
relate plant height), but we will get to that later.

We also ignore the effect of temperature on phenology (although this will be
added in)

The key is then to translate this underlying signal into the way a model
activates the plant it simulates and the allocation of leaf area to overall
plant leaf area index.

The hysteresis needs to be modelled and not ignored or excluded for the sake of
convenience.

Grass activation is negatively geared. The response to productivity is only
positive with the availability of adequate moisture, without which their is a
decay to zero.

As well as variance in the period of the signal, the available moisture may
have a greater impact upon the amplitude of leaf cover (\emph{greeness}).
